\documentclass{beamer}
\usepackage[utf8]{inputenc}
\usepackage{graphicx}

\title[Número pi]{Aproximación del número pi}
\author[Rebeka]{Rebeka Luis Hernández}
\date[25.04.14]{25 de Abril de 2014}

\begin{document}

\begin{frame}
 
  %\includegraphics[width=0.15\textwidth]{img/ullesc}
  \hspace*{7.0cm}
  %\includegraphics[width=0.16\textwidth]{img/fmatesc}
  \titlepage

  \begin{small}
    \begin{center}
     Facultad de Matemáticas \\
     Universidad de La Laguna
    \end{center}
  \end{small}

\end{frame}

\begin{frame}
  \frametitle{Índice}
  \tableofcontents[pausesections]
\end{frame}

\section{Primera Sección}
\section{Segunda Sección}
\section{Tercera Sección}

\begin{frame}

\frametitle{Historia del número pi}

A lo largo de la historia han sido muchas las formas utilizadas por el ser humanos para calcular aproximaciones 
cada vez más exactas del número $\pi$. El número $\pi$ es el cociente entre la longitud de una circunferencia 
cualquiera y el diámetro de la misma. Ludolph van Ceulen (1540-1610), matemático alemán profesor de la 
Universidad de Leiden en Holanda, se pasó buena parte de su vida calculandolos primeros 35 decimales de $\pi$.
Calcularemos una aproximación del número $\pi$ mediante la siguiente fórmula:

\end{frame}

\begin{frame}

\frametitle{Cálculo del número pi}

\begin{block}{}
Para realizar la aproximación del número pi es necesarios seguir una serie de pasos:
  \begin{itemize}
  \item
  Calculo del extremo de los subintervalos.
  \pause

  \item
  Calculo del punto $x_i$.
  \pause

  \item
  El valor de la función de aproximación de $pi$.

  \end{itemize}
\end{block}

\end{frame}

\begin{frame}
\frametitle{Fórmulas}
Se puede calcular mediante integración:

$\int_{0}^{1} \! \frac{4}{1+x^2}\, dx = 4(atan(1) -atan(0)) = \pi $


O mediante la regla del punto medio:


$\pi \approx \frac{1}{n} \sum\limits_{i=1}^{n}f(x_i)\,$,
con $f(x) = \frac{4}{(1+x^2)}\,$,
$x_i = \frac{i - \frac{1}{2}}{n}$,
para $i = , \dots, n$

Area del circulo $a = \pi \times r^2$



\end{frame}

\section{Bibliografía}
%++++++++++++++++++++++++++++++++++++++++++++++++++++++++++++++++++++++++++++++
\begin{frame}
  \frametitle{Bibliografía}

  \begin{thebibliography}{10}

    \beamertemplatebookbibitems
    \bibitem[Python]{tutorial}
    Tutorial Python.
    {\small $http://campusvirtual.ull.es/1314/course/view.php?id=1447$}

    \beamertemplatebookbibitems
    \bibitem[Beamer]{tutorial}
    Tutorial Beamer.
    {\small $http://campusvirtual.ull.es/1314/course/view.php?id=1447$}

    \beamertemplatebookbibitems
    \bibitem[URL: CTAN]{latex}
    CTAN. {\small $http://www.ctan.org/$}

  \end{thebibliography}
\end{frame}


\end{document}